% ═══════════════════════════════════════════════════════════════════════════
% TP53 VUS Reclassification — Scientific Manuscript
% Compile: pdflatex manuscript.tex && pdflatex manuscript.tex
% (Two passes to resolve cross-references; no bibtex needed.)
% ═══════════════════════════════════════════════════════════════════════════
\documentclass[11pt,a4paper,twocolumn]{article}

% ── Packages ──────────────────────────────────────────────────────────────
\usepackage[utf8]{inputenc}
\usepackage[T1]{fontenc}
\usepackage{mathptmx}                    % Times Roman
\usepackage[margin=2cm,columnsep=0.8cm]{geometry}
\usepackage{graphicx}
\usepackage{booktabs}
\usepackage{amsmath,amssymb}
\usepackage{siunitx}
\usepackage[hidelinks]{hyperref}
\usepackage{xcolor}
\usepackage{float}
\usepackage{placeins}                   % \FloatBarrier
\usepackage{caption}
\usepackage{subcaption}
\usepackage{natbib}
\usepackage{url}
\usepackage{enumitem}
\usepackage{microtype}
\usepackage{authblk}
\usepackage{lineno}
\usepackage{balance}

\linenumbers
\bibliographystyle{unsrtnat}

\graphicspath{{figures/main/}{figures/supplementary/}{../figures/main/}{../figures/supplementary/}}

\captionsetup{font=small,labelfont=bf}
\setlength{\parskip}{0.3em}

% ── Float tuning (reduce blank pages) ────────────────────────────────────
\renewcommand{\topfraction}{0.9}
\renewcommand{\bottomfraction}{0.9}
\renewcommand{\textfraction}{0.1}
\renewcommand{\floatpagefraction}{0.8}
\renewcommand{\dbltopfraction}{0.9}
\renewcommand{\dblfloatpagefraction}{0.8}
\setcounter{topnumber}{4}
\setcounter{bottomnumber}{4}
\setcounter{totalnumber}{8}
\setcounter{dbltopnumber}{4}

% ── Title ─────────────────────────────────────────────────────────────────
\title{%
  \textbf{Ensemble AI Screening of 1{,}211 TP53 Variants of Uncertain Significance\\
  via ESM-2 and AlphaMissense}%
}

\author{Mahad Asif}
\affil{9th Grade Computer Science Student}

\date{}

% ══════════════════════════════════════════════════════════════════════════
\begin{document}
\maketitle
\pagestyle{plain}

% ── Abstract ──────────────────────────────────────────────────────────────
\begin{abstract}
\noindent
\textbf{Background.}
TP53 is the most frequently mutated gene in human cancer, yet over 1{,}200
missense variants in ClinVar remain classified as Variants of Uncertain
Significance (VUS), limiting their clinical utility in precision oncology.

\textbf{Methods.}
We applied two orthogonal deep-learning models---Meta's ESM-2
(\texttt{esm2\_t33\_650M\_UR50D}; 650\,M parameters) and DeepMind's
AlphaMissense---to screen 1{,}211 TP53 ClinVar VUS.  Each variant was scored
by ESM-2 and cross-referenced against AlphaMissense predictions (UniProt
P04637), yielding 1{,}199 successfully matched variants.  Structural validation was performed against PDB 1TUP
(Cho et al., 1994).

\textbf{Results.}
The two models showed strong anti-correlation (Pearson $r = -0.706$;
Spearman $\rho = -0.714$), supporting complementary predictive capacity.
A total of 349 VUS were flagged as high-risk by both models (ESM-2 log-likelihood
ratio $\leq -4.0$ and AlphaMissense pathogenicity $> 0.564$), while 438
were concordantly classified as low-risk.  We highlight five variants in the
DNA-binding domain (L257R, V157D, R248P, C176R, R280I) with extreme
concordant scores (ESM-2 LLR $\leq -12.5$; AlphaMissense $\geq 0.99$)
and structurally validated damage mechanisms including loss of DNA contact,
zinc coordination abolishment, and hydrophobic core disruption.

\textbf{Conclusions.}
Ensemble AI scoring can systematically reclassify TP53 VUS at scale.
The 349 high-confidence pathogenic variants identified here warrant
prioritized functional validation and may inform germline testing
guidelines in Li--Fraumeni syndrome and somatic profiling in clinical oncology.

\medskip
\noindent\textbf{Keywords:}
TP53, variants of uncertain significance, ESM-2, AlphaMissense,
pathogenicity prediction, precision oncology, protein language model
\end{abstract}

% ══════════════════════════════════════════════════════════════════════════
\section{Introduction}
% ══════════════════════════════════════════════════════════════════════════

The \textit{TP53} gene encodes the tumour protein p53, a transcription factor
widely characterised as the ``Guardian of the Genome'' for its central role in
maintaining genomic integrity \citep{lane1992}.  In response to genotoxic
stress, p53 activates cell-cycle arrest, DNA repair, senescence, and apoptosis
through sequence-specific DNA binding at target promoters
\citep{vogelstein2000,levine2009}.  Loss-of-function mutations in TP53 are
the single most common genetic alteration across all human cancers, observed in
over 50\% of solid tumours \citep{kandoth2013}, and germline TP53 mutations are
the molecular basis of Li--Fraumeni syndrome, a hereditary cancer predisposition
disorder conferring near-complete lifetime cancer penetrance \citep{malkin1990}.

The clinical interpretation of TP53 missense variants has been greatly advanced
by large-scale repositories such as ClinVar \citep{landrum2018} and the IARC
TP53 Database \citep{bouaoun2016}.  However, a substantial fraction of observed
variants---over 1{,}200 unique missense substitutions in ClinVar at the time of
this analysis---remain classified as Variants of Uncertain Significance (VUS).
The VUS designation creates a ``diagnostic grey zone'' for clinicians: these
variants cannot be used to guide treatment decisions, genetic counselling, or
cascade family testing, even when the variant resides in a functionally critical
domain \citep{richards2015}.

The bottleneck in VUS resolution is fundamentally one of evidence accumulation.
Under the ACMG/AMP framework \citep{richards2015}, reclassification requires
convergent evidence from population frequency data, \textit{in silico}
prediction, functional assays, co-segregation studies, and \textit{de novo}
occurrence.  For rare variants observed only once or twice in clinical
databases, such evidence may never accrue through observation alone.
Large-scale functional assays, such as the saturation mutagenesis screen of
TP53 by \citet{giacomelli2018}, have made important contributions but remain
labour-intensive and costly.

Recent advances in protein language models (pLMs) and structure-based
pathogenicity predictors offer a scalable complement to experimental
approaches.  Meta's ESM-2 \citep{lin2023} is a transformer-based protein
language model trained on 250 million sequences from UniRef, capable of
zero-shot variant effect prediction through log-likelihood ratios.  DeepMind's
AlphaMissense \citep{cheng2023} combines AlphaFold2 structural features with
sequence context to classify all possible human missense variants, achieving
state-of-the-art performance on ClinVar benchmarks.

Critically, ESM-2 and AlphaMissense derive their predictions from orthogonal
information sources: ESM-2 operates purely from evolutionary sequence
conservation patterns, whereas AlphaMissense integrates structural features
from AlphaFold2 with population frequency priors.  An ensemble approach
leveraging both models therefore provides complementary evidence that can
strengthen confidence in pathogenicity assignments beyond what either model
achieves alone.

In this study, we systematically apply ESM-2 and AlphaMissense to
1{,}211 TP53 ClinVar VUS, cross-referencing both models' predictions to
identify high-confidence pathogenic candidates.  We identify 349
variants with concordant high-risk scores, highlight five candidates with
extreme pathogenicity signals and structurally validated damage mechanisms, and
discuss the implications of these findings for clinical variant
reclassification in oncology.

% ══════════════════════════════════════════════════════════════════════════
\section{Materials and Methods}
% ══════════════════════════════════════════════════════════════════════════

% ── 2.1 ──────────────────────────────────────────────────────────────────
\subsection{Variant Ascertainment}

TP53 missense variants classified as ``Uncertain Significance'' were retrieved
from NCBI ClinVar \citep{landrum2018} via the Entrez E-utilities API.  Variants
were filtered to retain only single-nucleotide missense substitutions mapped to
the canonical TP53 protein isoform (RefSeq NP\_000537.3; UniProt P04637; 393
amino acids).  After deduplication by protein-level change in HGVS notation,
1{,}211 unique VUS were retained.

% ── 2.2 ──────────────────────────────────────────────────────────────────
\subsection{ESM-2 Variant Effect Prediction}

Variant effect scores were computed using Meta's ESM-2 protein language model
(\texttt{esm2\_t33\_650M\_UR50D}; 650 million parameters) \citep{lin2023},
accessed via the HuggingFace Transformers library.  For each variant, the
wild-type TP53 sequence was passed through the model, and the log-likelihood
ratio (LLR) was calculated as:

\begin{equation}
  \text{LLR} = \log P(x_{\text{mut}} \mid \mathbf{x}_{\setminus i})
             - \log P(x_{\text{wt}}  \mid \mathbf{x}_{\setminus i})
  \label{eq:llr}
\end{equation}

\noindent where $P(x \mid \mathbf{x}_{\setminus i})$ denotes the model's
predicted probability for amino acid $x$ at position $i$, conditioned on the
full sequence context.  Negative LLR values indicate that the mutation is
disfavoured by evolutionary constraints captured in the model.  Scores were
stratified into four tiers: strongly damaging (LLR $\leq -4.0$), likely
damaging ($-4.0 <$ LLR $\leq -2.0$), possibly damaging ($-2.0 <$ LLR $\leq
-0.5$), and likely neutral (LLR $> -0.5$).

All 1{,}211 VUS were scored in batch mode.  Computation was performed on a
consumer laptop equipped with GPU acceleration, with an average throughput of
approximately 3 variants per second.  Twelve variants where the ClinVar
reference amino acid did not match the UniProt wild-type sequence at the
stated position were flagged as wild-type mismatches and excluded from
downstream analysis.

% ── 2.3 ──────────────────────────────────────────────────────────────────
\subsection{AlphaMissense Pathogenicity Scores}

Precomputed AlphaMissense pathogenicity scores \citep{cheng2023} for all
possible single-amino-acid substitutions in the human proteome were downloaded
from the Zenodo repository (record 10813168; file size $\sim$1.12\,GB
compressed).  The dataset was stream-filtered for TP53 (UniProt P04637),
yielding 7{,}467 scored substitutions spanning all 393 residue positions.
AlphaMissense scores range from 0 to 1, with the recommended classification
thresholds of pathogenic ($>0.564$), ambiguous ($0.340$--$0.564$), and benign
($<0.340$) as defined by \citet{cheng2023}.

Of the 7{,}467 TP53 substitutions scored by AlphaMissense, 3{,}417 (45.8\%)
were classified as pathogenic, 763 (10.2\%) as ambiguous, and 3{,}287 (44.0\%)
as benign.

% ── 2.4 ──────────────────────────────────────────────────────────────────
\subsection{Cross-Referencing and Concordance Analysis}

ESM-2 and AlphaMissense scores were merged on the protein-level variant
identifier (e.g., ``L257R'').  Of the 1{,}211 ESM-2-scored VUS, 1{,}199
(99.0\%) were successfully matched to an AlphaMissense prediction.  Concordance
between the two models was assessed using Pearson and Spearman correlation
coefficients.  Variants were classified into four quadrants based on dual
thresholds: high-risk concordant (ESM-2 LLR $\leq -4.0$ \textit{and}
AlphaMissense $> 0.564$), low-risk concordant (ESM-2 LLR $> -0.5$ \textit{and}
AlphaMissense $< 0.340$), and two discordant categories.

% ── 2.5 ──────────────────────────────────────────────────────────────────
\subsection{Structural Validation}

Structural context for the top-ranked variants was assessed using the
crystal structure of the p53 core domain--DNA complex (PDB: 1TUP; 2.2\,\AA{}
resolution) \citep{cho1994}.  This structure captures the p53 DNA-binding
domain (residues $\sim$94--312) of chain B bound sequence-specifically to a
21-bp DNA duplex (chains E and F), with a structural Zn$^{2+}$ ion coordinated
by Cys176, His179, Cys238, and Cys242.

Two complementary structural analysis approaches were employed:

\textit{Computational geometry analysis.}
Spatial context was computed with BioPython's \texttt{NeighborSearch}
module \citep{cock2009}, using a contact distance threshold of 4.0\,\AA{}.
For each variant site, we identified neighbouring protein residues within
the contact shell, DNA atoms within contact distance, and zinc ion proximity.
Results were visualised as three-dimensional scatter plots showing the
12\,\AA{} radius structural environment around each mutation site
(Supplementary Fig.~S1).

\textit{Publication-quality structural rendering.}
High-resolution renders were generated using PyMOL (open-source version 3.1.0)
\citep{pymol} in headless mode.  For each variant, the wild-type residue was
displayed alongside a computationally modelled mutant rotamer (applied via
PyMOL's mutagenesis wizard) to visualise steric and chemical differences.
All renders were ray-traced at 2{,}400 $\times$ 2{,}400 pixels with
antialiasing level~4.  An initial set of single-residue PyMOL renders was
also generated from a scripted \texttt{.pml} pipeline (Supplementary Fig.~S2).

% ── 2.6 ──────────────────────────────────────────────────────────────────
\subsection{Computational Environment}

Primary analysis was performed on a consumer laptop running Windows.  ESM-2
inference utilised local GPU acceleration via PyTorch with CUDA.  AlphaMissense
data were obtained from Google Cloud Storage / Zenodo.  All analysis scripts
were implemented in Python~3, using BioPython \citep{cock2009}, PyTorch,
HuggingFace Transformers, NumPy, and Matplotlib.  Structural rendering
employed PyMOL 3.1.0 via the \texttt{micromamba} package manager.

% ══════════════════════════════════════════════════════════════════════════
\section{Results}
% ══════════════════════════════════════════════════════════════════════════

% ── 3.1 ──────────────────────────────────────────────────────────────────
\subsection{ESM-2 Score Distribution across TP53 VUS}

ESM-2 log-likelihood ratios for the 1{,}199 matched VUS ranged from $-13.88$
to $+4.21$ (Fig.~\ref{fig:llr_dist}).  The distribution was left-skewed,
consistent with the DNA-binding domain harbouring a disproportionate fraction
of damaging variants.  Of the 1{,}199 scored variants, 380 (31.7\%) were
classified as strongly damaging (LLR $\leq -4.0$), 232 (19.3\%) as likely
damaging, 282 (23.5\%) as possibly damaging, and 305 (25.4\%) as likely
neutral.

\begin{figure}[!htb]
  \centering
  \includegraphics[width=\columnwidth]{fig1_llr_distribution.png}
  \caption{Distribution of ESM-2 log-likelihood ratio (LLR) scores across
  1{,}199 TP53 VUS.  Dashed vertical lines indicate classification thresholds.
  Variants with LLR $\leq -4.0$ are designated strongly damaging.}
  \label{fig:llr_dist}
\end{figure}

% ── 3.2 ──────────────────────────────────────────────────────────────────
\subsection{Model Concordance}

ESM-2 LLR and AlphaMissense pathogenicity scores showed a strong negative
correlation (Pearson $r = -0.706$; Spearman $\rho = -0.714$;
Fig.~\ref{fig:concordance}), indicating that variants scored as highly
damaging by ESM-2 (more negative LLR) were independently scored as highly
pathogenic by AlphaMissense (score approaching 1.0).  The anti-correlation
is expected given the inverse directionality of the two scoring scales.

\begin{figure}[!htb]
  \centering
  \includegraphics[width=\columnwidth]{fig2_model_concordance.png}
  \caption{Scatter plot of ESM-2 LLR versus AlphaMissense pathogenicity score
  for 1{,}199 matched TP53 VUS.  The strong anti-correlation (Pearson
  $r = -0.706$; Spearman $\rho = -0.714$) supports complementary predictive
  capacity between the sequence-based and structure-based models.}
  \label{fig:concordance}
\end{figure}

Applying dual thresholds, 349 variants (29.1\%) were classified as high-risk
by both models, and 438 (36.5\%) were classified as low-risk by both models.
The remaining 412 variants (34.4\%) showed discordant classifications, occupying
the ambiguous region where additional evidence is needed.

% ── 3.3 ──────────────────────────────────────────────────────────────────
\subsection{Domain-Level Enrichment}

Variant pathogenicity scores were non-uniformly distributed across TP53
functional domains (Fig.~\ref{fig:domains}).  The DNA-binding domain
(residues 102--292) showed significantly more negative mean ESM-2 LLR scores
than the transactivation domain (residues 1--92) or the tetramerization domain
(residues 325--356), consistent with stronger evolutionary constraint on the
DNA-binding interface.  All five top-ranked variants mapped to the DNA-binding
domain.

\begin{figure}[!htb]
  \centering
  \includegraphics[width=\columnwidth]{fig3_domain_enrichment.png}
  \caption{ESM-2 LLR score distribution by TP53 functional domain.  The
  DNA-binding domain (residues 102--292) harbours the most severely scored
  variants, consistent with its critical role in tumour suppression.}
  \label{fig:domains}
\end{figure}

% ── 3.4 ──────────────────────────────────────────────────────────────────
\subsection{Top Five Candidate Pathogenic Variants}

Table~\ref{tab:top5} presents the five VUS with the most extreme concordant
pathogenicity scores.  All five reside in the DNA-binding domain of p53 and
exhibit structurally validated damage mechanisms.

\begin{table*}[!htb]
  \centering
  \caption{Top five TP53 VUS with extreme concordant pathogenicity scores.
  All variants map to the DNA-binding domain (residues 102--292) of PDB 1TUP,
  chain B.  ESM-2 LLR values below $-4.0$ indicate strong damage;
  AlphaMissense scores above 0.564 indicate pathogenicity.}
  \label{tab:top5}
  \small
  \begin{tabular}{@{}llccccl@{}}
    \toprule
    \textbf{Variant} & \textbf{HGVS} & \textbf{Residue} &
    \textbf{ESM-2 LLR} & \textbf{AM Score} & \textbf{ClinVar ID} &
    \textbf{Structural Mechanism} \\
    \midrule
    L257R & p.Leu257Arg & 257 & $-13.88$ & 0.9938 & 142134 &
      Hydrophobic core disruption \\
    V157D & p.Val157Asp & 157 & $-13.69$ & 0.9992 & 482231 &
      Hydrophobic core disruption \\
    R248P & p.Arg248Pro & 248 & $-12.63$ & 0.9994 & 237954 &
      DNA minor groove contact lost \\
    C176R & p.Cys176Arg & 176 & $-12.53$ & 0.9999 & 376573 &
      Zinc coordination abolished \\
    R280I & p.Arg280Ile & 280 & $-12.47$ & 0.9996 & 161517 &
      DNA major groove contact lost \\
    \bottomrule
  \end{tabular}
\end{table*}

% ── 3.4.1 ────────────────────────────────────────────────────────────────
\subsubsection{L257R (p.Leu257Arg): Hydrophobic Core Disruption}

Leucine 257 is buried within the hydrophobic $\beta$-sandwich core of the p53
DNA-binding domain.  The substitution to arginine introduces a positively
charged, bulky side chain into a tightly packed nonpolar environment.  This
variant received the most extreme ESM-2 LLR of $-13.88$ and an AlphaMissense
score of 0.9938, consistent with severe destabilisation of the protein fold
(Fig.~\ref{fig:renders}a).

% ── 3.4.2 ────────────────────────────────────────────────────────────────
\subsubsection{V157D (p.Val157Asp): Hydrophobic Core Disruption}

Valine 157 occupies a $\beta$-strand within the interior of the same
$\beta$-sandwich.  Substitution to aspartate introduces a negative charge and
a shorter side chain, creating both electrostatic repulsion and a packing
cavity.  The ESM-2 LLR of $-13.69$ and AlphaMissense score of 0.9992 both
indicate near-certain pathogenicity (Fig.~\ref{fig:renders}b).

% ── 3.4.3 ────────────────────────────────────────────────────────────────
\subsubsection{R248P (p.Arg248Pro): Loss of DNA Minor Groove Contact}

Arginine 248 is one of the most frequently mutated residues in human cancer.
In the wild-type structure, R248 inserts directly into the DNA minor groove at
a distance of approximately 3.4\,\AA{}, forming critical hydrogen bonds with
the DNA backbone.  Proline at this position eliminates all hydrogen-bonding
capacity and introduces a rigid kink in the polypeptide backbone.  Both
models scored this variant at extreme levels (LLR $= -12.63$; AM $= 0.9994$)
(Fig.~\ref{fig:renders}c).

% ── 3.4.4 ────────────────────────────────────────────────────────────────
\subsubsection{C176R (p.Cys176Arg): Zinc Coordination Abolished}

Cysteine 176 is one of four residues (C176, H179, C238, C242) that coordinate
the structural Zn$^{2+}$ ion essential for the folding of the L2 and L3 loops
of the DNA-binding domain.  The thiolate side chain of cysteine provides a
ligand to zinc at a distance of approximately 2.3\,\AA{}.  Arginine cannot
coordinate zinc, and its introduction is predicted to collapse the local loop
scaffold.  AlphaMissense assigned the highest score in our dataset (0.9999)
to this variant (Fig.~\ref{fig:renders}d).

% ── 3.4.5 ────────────────────────────────────────────────────────────────
\subsubsection{R280I (p.Arg280Ile): Loss of DNA Major Groove Contact}

Arginine 280 forms a direct hydrogen bond with a guanine base in the DNA
major groove at a distance of approximately 2.8\,\AA{}.  This contact is
essential for sequence-specific DNA recognition.  Substitution to isoleucine,
a hydrophobic residue with no hydrogen-bonding capacity, abolishes this
interaction entirely (LLR $= -12.47$; AM $= 0.9996$) (Fig.~\ref{fig:renders}e).

% ── Figure 4: Publication structural renders ─────────────────────────────

\begin{figure*}[!htb]
  \centering
  \begin{subfigure}[t]{0.32\textwidth}
    \includegraphics[width=\textwidth]{fig4a_L257R_structural.png}
    \caption{L257R}
  \end{subfigure}\hfill
  \begin{subfigure}[t]{0.32\textwidth}
    \includegraphics[width=\textwidth]{fig4b_V157D_structural.png}
    \caption{V157D}
  \end{subfigure}\hfill
  \begin{subfigure}[t]{0.32\textwidth}
    \includegraphics[width=\textwidth]{fig4c_R248P_structural.png}
    \caption{R248P}
  \end{subfigure}

  \medskip
  \begin{subfigure}[t]{0.32\textwidth}
    \includegraphics[width=\textwidth]{fig4d_C176R_structural.png}
    \caption{C176R}
  \end{subfigure}\hfill
  \begin{subfigure}[t]{0.32\textwidth}
    \includegraphics[width=\textwidth]{fig4e_R280I_structural.png}
    \caption{R280I}
  \end{subfigure}\hfill
  \begin{subfigure}[t]{0.32\textwidth}
    \includegraphics[width=\textwidth]{fig4f_overview_structural.png}
    \caption{Overview}
  \end{subfigure}

  \caption{Publication-quality structural renders of the top five TP53 VUS
  (PDB 1TUP, chain B).  Wild-type residues are shown as green sticks;
  computationally modelled mutant rotamers (PyMOL mutagenesis wizard) as
  salmon sticks.  Neighbouring residues within 4.0\,\AA{} are shown as
  light-blue sticks.  DNA polar contacts are indicated by red dashes; zinc
  coordination by slate dashes.
  (a)~L257R: charged arginine disrupts the hydrophobic $\beta$-sandwich core.
  (b)~V157D: aspartate introduces charge and a packing void.
  (c)~R248P: proline eliminates DNA minor groove contact.
  (d)~C176R: arginine abolishes Zn$^{2+}$ coordination.
  (e)~R280I: isoleucine abolishes DNA major groove hydrogen bond.
  (f)~Overview of all five sites (coloured spheres) mapped onto the
  p53--DNA complex.  Protein shown as gray cartoon; DNA as orange cartoon;
  Zn$^{2+}$ ions as slate spheres.
  All panels ray-traced at 2{,}400$\times$2{,}400 pixels, 300\,DPI.}
  \label{fig:renders}
\end{figure*}

% ══════════════════════════════════════════════════════════════════════════
\section{Discussion}
% ══════════════════════════════════════════════════════════════════════════

% ── 4.1 ──────────────────────────────────────────────────────────────────
\subsection{Concordant AI Prediction as Evidence for Reclassification}

The strong anti-correlation between ESM-2 and AlphaMissense predictions
($r = -0.706$; $\rho = -0.714$) is noteworthy because the two models were
trained on fundamentally different data representations.  ESM-2 is a pure
sequence model that learns evolutionary constraints from 250 million protein
sequences without any explicit structural information \citep{lin2023}.
AlphaMissense, by contrast, incorporates AlphaFold2-derived structural features
alongside sequence context and population frequency data \citep{cheng2023}.
The convergence of these orthogonal approaches on the same set of high-risk
variants provides a form of computational triangulation analogous to the
convergent evidence required by the ACMG/AMP framework \citep{richards2015}.

The 349 variants flagged as high-risk by both models represent 29.1\% of the
matched ClinVar VUS.  Under current clinical guidelines, none of these variants
are actionable.  Our analysis suggests that a substantial subset---particularly
those with extreme scores such as the five candidates highlighted here---carry
sufficient \textit{in silico} evidence to support provisional reclassification,
pending functional confirmation.

% ── 4.2 ──────────────────────────────────────────────────────────────────
\subsection{Structural Basis of Predicted Pathogenicity}

The five top-ranked variants illustrate three distinct molecular mechanisms
of p53 loss of function, each corroborated by structural analysis of PDB 1TUP
and the three-dimensional spatial context computed via BioPython
(Supplementary Fig.~S1):

\begin{enumerate}[nosep]
  \item \textbf{Direct DNA contact loss} (R248P, R280I): These residues form
    hydrogen bonds with DNA bases or backbone atoms in the p53 response
    element.  R248 is a known mutational hotspot in cancer (R248W, R248Q are
    among the six most common TP53 mutations) \citep{bouaoun2016}, but the
    R248P substitution---which introduces a conformationally rigid proline---has
    not been previously classified as pathogenic in ClinVar despite affecting
    the same critical contact residue.

  \item \textbf{Zinc coordination abolishment} (C176R): The structural
    Zn$^{2+}$ ion is essential for the folding and stability of the L2--L3
    loop region that forms part of the DNA-binding surface
    \citep{cho1994,bullock2000}.  Loss of even one zinc ligand is expected
    to destabilise the entire loop scaffold.

  \item \textbf{Hydrophobic core disruption} (L257R, V157D): Introduction of
    charged residues into the buried $\beta$-sandwich core is a well-established
    mechanism of p53 thermodynamic destabilisation \citep{bullock2000}.  Such
    mutations reduce the melting temperature of the DNA-binding domain and
    accelerate unfolding at physiological temperature.
\end{enumerate}

The structural consistency between the AI predictions and the known
three-dimensional architecture of the p53--DNA complex provides a mechanistic
rationale for the extreme scores observed, and supports the biological
plausibility of the reclassification.  Two complementary structural
visualisation approaches---BioPython-based three-dimensional context plots
(Supplementary Fig.~S1) and PyMOL ray-traced renders at multiple levels of
detail (Fig.~\ref{fig:renders}; Supplementary Fig.~S2)---confirm the
spatial relationships described above.

% ── 4.3 ──────────────────────────────────────────────────────────────────
\subsection{Implications for Personalised Oncology}

The clinical impact of resolving TP53 VUS extends across multiple domains of
cancer care:

\textbf{Germline testing in Li--Fraumeni syndrome.}
Li--Fraumeni syndrome (LFS) is diagnosed by the identification of a pathogenic
germline TP53 variant.  Individuals with LFS face a cumulative cancer risk
exceeding 90\% by age 60, and benefit from intensive surveillance protocols
including annual whole-body MRI \citep{villani2016}.  When a TP53 variant
detected on germline panel testing is classified as VUS, the patient and their
family members cannot be offered definitive risk stratification.
Reclassification of high-confidence VUS to likely pathogenic would directly
enable cascade testing and early surveillance.

\textbf{Somatic tumour profiling.}
TP53 mutational status is a key biomarker in haematological malignancies,
where it predicts resistance to chemoimmunotherapy in chronic lymphocytic
leukaemia \citep{zenz2010} and adverse prognosis in myelodysplastic syndromes
\citep{bejar2011}.  In solid tumours, TP53 status informs prognosis and,
increasingly, therapy selection in the context of synthetic lethality
approaches.  Resolving VUS enables more precise molecular stratification.

\textbf{Emerging p53-targeted therapies.}
A new generation of therapeutics aims to restore or stabilise mutant p53
function.  Small molecules such as APR-246 (eprenetapopt) and
PC14586 (rezatapopt) have entered clinical trials for tumours harbouring
specific TP53 mutations \citep{chen2021}.  Accurate classification of
TP53 variants is a prerequisite for patient selection in these trials.
Variants that destabilise the protein fold (L257R, V157D) may respond to
fold-stabilising compounds, whereas those that abolish DNA contact
(R248P, R280I) may require distinct therapeutic strategies.

% ── 4.4 ──────────────────────────────────────────────────────────────────
\subsection{Limitations}

Several limitations should be noted.  First, \textit{in silico} predictions,
regardless of model concordance, do not constitute functional evidence under
ACMG/AMP criteria (PP3) and cannot alone support a pathogenic classification
beyond ``supporting'' evidence strength.  Second, our structural analysis is
based on a single crystal structure (PDB 1TUP) that captures only one
conformational state of the p53 tetramer--DNA complex; dynamic effects and
post-translational modifications are not represented.  Third, ESM-2 LLR
thresholds ($\leq -4.0$) and AlphaMissense cutoffs ($>0.564$) were adopted
from the original publications and have not been independently calibrated on
a TP53-specific truth set.  Finally, the clinical significance of the 349
high-risk VUS should be regarded as provisional until orthogonal experimental
validation---such as yeast-based functional assays, thermal stability
measurements, or DNA-binding electrophoretic mobility shift assays---is
completed.

% ── 4.5 ──────────────────────────────────────────────────────────────────
\subsection{Future Directions}

This work motivates several follow-up investigations.  The 349 concordant
high-risk variants represent a prioritised set for experimental functional
validation, potentially through high-throughput approaches such as
multiplexed assays of variant effect (MAVEs) \citep{findlay2018}.  Integration
of additional \textit{in silico} tools---including EVE \citep{frazer2021},
REVEL \citep{ioannidis2016}, and molecular dynamics simulations---could further
refine the confidence tiers.  Longitudinal tracking of these variants in ClinVar
will reveal whether independent clinical evidence eventually converges with
the AI predictions presented here, providing a natural validation of the
ensemble approach.

% ══════════════════════════════════════════════════════════════════════════
\section{Conclusions}
% ══════════════════════════════════════════════════════════════════════════

We demonstrate that ensemble AI scoring using ESM-2 and AlphaMissense can
systematically identify high-confidence pathogenic variants among the 1{,}211
TP53 VUS currently in ClinVar.  The 349 concordant high-risk variants---and
particularly the five extreme candidates (L257R, V157D, R248P, C176R,
R280I)---exhibit both computational and structural hallmarks of loss of
function.  These findings support the integration of orthogonal AI models as
a scalable component of variant classification pipelines, with direct
implications for germline testing, somatic profiling, and patient selection
for emerging p53-targeted therapies in precision oncology.

% ══════════════════════════════════════════════════════════════════════════
\section*{Data Availability}
% ══════════════════════════════════════════════════════════════════════════

All analysis scripts, scored variant data, and structural renders are available
in the project repository.  ClinVar data were accessed via NCBI Entrez.
AlphaMissense proteome-wide predictions are available from Zenodo (record
10813168).  The PDB structure 1TUP is available from the RCSB Protein Data
Bank.

% ══════════════════════════════════════════════════════════════════════════
\section*{Acknowledgements}
% ══════════════════════════════════════════════════════════════════════════

The author thanks the developers of ESM-2 (Meta AI), AlphaMissense (Google
DeepMind), ClinVar (NCBI), BioPython, and PyMOL for making their tools and
data freely accessible.  The 1TUP crystal structure was obtained from the
RCSB Protein Data Bank.  This project was developed as an independent
computational biology research effort.

% ══════════════════════════════════════════════════════════════════════════
\section*{Author Contact Information}
% ══════════════════════════════════════════════════════════════════════════

\begin{itemize}[nosep,leftmargin=1.5em]
  \item \textbf{Author:} Mahad Asif
  \item \textbf{Email:} \href{mailto:mahaddevx@gmail.com}{mahaddevx@gmail.com}
  \item \textbf{GitHub:} \href{https://github.com/mahaddev-x}{github.com/mahaddev-x}
  \item \textbf{Project Repository:} \href{https://github.com/mahaddev-x/TP53-VUS-Predict}{github.com/mahaddev-x/TP53-VUS-Predict}
\end{itemize}

\medskip
\noindent
All source code, scored variant tables, structural analysis outputs, and
PyMOL render scripts used in this study are publicly available in the
project repository above.

% ══════════════════════════════════════════════════════════════════════════
% References (two-column, before supplementary)
% ══════════════════════════════════════════════════════════════════════════

\begin{thebibliography}{30}

\bibitem[Bejar et~al.(2011)]{bejar2011}
Bejar, R., Stevenson, K., Abdel-Wahab, O., et~al.
\newblock Clinical effect of point mutations in myelodysplastic syndromes.
\newblock \textit{N.~Engl.~J.~Med.}, 364(26):2496--2506, 2011.

\bibitem[Bouaoun et~al.(2016)]{bouaoun2016}
Bouaoun, L., Sonber, D., Ardin, M., et~al.
\newblock TP53 variations in human cancers: new lessons from the IARC TP53
  Database and genomics data.
\newblock \textit{Hum.~Mutat.}, 37(9):865--876, 2016.

\bibitem[Bullock et~al.(2000)]{bullock2000}
Bullock, A.~N., Henckel, J. \& Fersht, A.~R.
\newblock Quantitative analysis of residual folding and DNA binding in mutant
  p53 core domain: definition of mutant states for rescue in cancer therapy.
\newblock \textit{Oncogene}, 19(10):1245--1256, 2000.

\bibitem[Chen et~al.(2021)]{chen2021}
Chen, S., Wu, J.-L., Liang, Y., et~al.
\newblock Small molecule therapeutics for TP53-mutant cancers.
\newblock \textit{Trends Pharmacol.~Sci.}, 42(12):1049--1062, 2021.

\bibitem[Cheng et~al.(2023)]{cheng2023}
Cheng, J., Novati, G., Pan, J., et~al.
\newblock Accurate proteome-wide missense variant effect prediction with
  AlphaMissense.
\newblock \textit{Science}, 381(6664):eadg7492, 2023.

\bibitem[Cho et~al.(1994)]{cho1994}
Cho, Y., Gorina, S., Jeffrey, P.~D. \& Pavletich, N.~P.
\newblock Crystal structure of a p53 tumor suppressor--DNA complex:
  understanding tumorigenic mutations.
\newblock \textit{Science}, 265(5170):346--355, 1994.

\bibitem[Cock et~al.(2009)]{cock2009}
Cock, P. J.~A., Antao, T., Chang, J.~T., et~al.
\newblock Biopython: freely available Python tools for computational molecular
  biology and bioinformatics.
\newblock \textit{Bioinformatics}, 25(11):1422--1423, 2009.

\bibitem[Findlay et~al.(2018)]{findlay2018}
Findlay, G.~M., Daza, R.~M., Martin, B., et~al.
\newblock Accurate classification of BRCA1 variants with saturation genome
  editing.
\newblock \textit{Nature}, 562(7726):217--222, 2018.

\bibitem[Frazer et~al.(2021)]{frazer2021}
Frazer, J., Notin, P., Dias, M., et~al.
\newblock Disease variant prediction with deep generative models of
  evolutionary data.
\newblock \textit{Nature}, 599(7883):91--95, 2021.

\bibitem[Giacomelli et~al.(2018)]{giacomelli2018}
Giacomelli, A.~O., Yang, X., Lintber, R.~E., et~al.
\newblock Mutational processes shape the landscape of TP53 mutations in human
  cancer.
\newblock \textit{Nat.~Genet.}, 50(10):1381--1387, 2018.

\bibitem[Ioannidis et~al.(2016)]{ioannidis2016}
Ioannidis, N.~M., Rothstein, J.~H., Pejaver, V., et~al.
\newblock REVEL: an ensemble method for predicting the pathogenicity of rare
  missense variants.
\newblock \textit{Am.~J.~Hum.~Genet.}, 99(4):877--885, 2016.

\bibitem[Kandoth et~al.(2013)]{kandoth2013}
Kandoth, C., McLellan, M.~D., Vandin, F., et~al.
\newblock Mutational landscape and significance across 12 major cancer types.
\newblock \textit{Nature}, 502(7471):333--339, 2013.

\bibitem[Landrum et~al.(2018)]{landrum2018}
Landrum, M.~J., Lee, J.~M., Benson, M., et~al.
\newblock ClinVar: improving access to variant interpretations and supporting
  evidence.
\newblock \textit{Nucleic Acids Res.}, 46(D1):D1062--D1067, 2018.

\bibitem[Lane(1992)]{lane1992}
Lane, D.~P.
\newblock p53, guardian of the genome.
\newblock \textit{Nature}, 358(6381):15--16, 1992.

\bibitem[Levine \& Oren(2009)]{levine2009}
Levine, A.~J. \& Oren, M.
\newblock The first 30 years of p53: growing ever more complex.
\newblock \textit{Nat.~Rev.~Cancer}, 9(10):749--758, 2009.

\bibitem[Lin et~al.(2023)]{lin2023}
Lin, Z., Akin, H., Rao, R., et~al.
\newblock Evolutionary-scale prediction of atomic-level protein structure with
  a language model.
\newblock \textit{Science}, 379(6637):1123--1130, 2023.

\bibitem[Malkin et~al.(1990)]{malkin1990}
Malkin, D., Li, F.~P., Strong, L.~C., et~al.
\newblock Germ line p53 mutations in a familial syndrome of breast cancer,
  sarcomas, and other neoplasms.
\newblock \textit{Science}, 250(4985):1233--1238, 1990.

\bibitem[{The PyMOL Molecular Graphics System}(2015)]{pymol}
{The PyMOL Molecular Graphics System}, Version 3.1, Schr\"odinger, LLC.

\bibitem[Richards et~al.(2015)]{richards2015}
Richards, S., Aziz, N., Bale, S., et~al.
\newblock Standards and guidelines for the interpretation of sequence variants:
  a joint consensus recommendation of the American College of Medical Genetics
  and Genomics and the Association for Molecular Pathology.
\newblock \textit{Genet.~Med.}, 17(5):405--424, 2015.

\bibitem[Villani et~al.(2016)]{villani2016}
Villani, A., Shore, A., Wasserman, J.~D., et~al.
\newblock Biochemical and imaging surveillance in germline TP53 mutation
  carriers with Li--Fraumeni syndrome: 11 year follow-up of a prospective
  observational study.
\newblock \textit{Lancet Oncol.}, 17(9):1295--1305, 2016.

\bibitem[Vogelstein et~al.(2000)]{vogelstein2000}
Vogelstein, B., Lane, D. \& Levine, A.~J.
\newblock Surfing the p53 network.
\newblock \textit{Nature}, 408(6810):307--310, 2000.

\bibitem[Zenz et~al.(2010)]{zenz2010}
Zenz, T., Eichhorst, B., Busch, R., et~al.
\newblock TP53 mutation and survival in chronic lymphocytic leukemia.
\newblock \textit{J.~Clin.~Oncol.}, 28(29):4473--4479, 2010.

\end{thebibliography}

% ══════════════════════════════════════════════════════════════════════════
% Supplementary Figures
% ══════════════════════════════════════════════════════════════════════════

\clearpage
\onecolumn
\pagestyle{plain}
\appendix
\renewcommand{\thefigure}{S\arabic{figure}}
\setcounter{figure}{0}

\section*{Supplementary Figures}

% ── Supplementary Figure S1: BioPython 3D structural context ─────────────

\begin{figure}[H]
  \centering
  \begin{subfigure}[t]{0.38\textwidth}
    \includegraphics[width=\textwidth]{figS1a_L257R_3d_context.png}
    \caption{L257R --- Core disruption}
  \end{subfigure}\hfill
  \begin{subfigure}[t]{0.38\textwidth}
    \includegraphics[width=\textwidth]{figS1b_V157D_3d_context.png}
    \caption{V157D --- Core disruption}
  \end{subfigure}

  \smallskip
  \begin{subfigure}[t]{0.38\textwidth}
    \includegraphics[width=\textwidth]{figS1c_R248P_3d_context.png}
    \caption{R248P --- Minor groove lost}
  \end{subfigure}\hfill
  \begin{subfigure}[t]{0.38\textwidth}
    \includegraphics[width=\textwidth]{figS1d_C176R_3d_context.png}
    \caption{C176R --- Zinc abolished}
  \end{subfigure}

  \smallskip
  \begin{subfigure}[t]{0.38\textwidth}
    \includegraphics[width=\textwidth]{figS1e_R280I_3d_context.png}
    \caption{R280I --- Major groove lost}
  \end{subfigure}\hfill
  \begin{subfigure}[t]{0.38\textwidth}
    \includegraphics[width=\textwidth]{figS1f_overview_3d_context.png}
    \caption{Overview --- All five sites}
  \end{subfigure}

  \caption{\textbf{Supplementary Figure~S1: Three-dimensional structural
  context of the top five TP53 VUS (BioPython analysis).}
  Each panel shows a 12\,\AA{} radius view centred on the variant
  residue (green) in PDB 1TUP chain~B.  Orange: DNA; light gray:
  C$\alpha$ backbone; cornflower blue: neighbours within 4.0\,\AA{};
  red diamonds: DNA contacts; slate: Zn$^{2+}$.
  (a)~L257R buried, no DNA contact.
  (b)~V157D buried in $\beta$-sandwich.
  (c)~R248P DNA minor groove contact.
  (d)~C176R Zn$^{2+}$ coordination.
  (e)~R280I DNA major groove contact.
  (f)~Overview of all five positions on chain~B.}
  \label{fig:supp_3d}
\end{figure}

% ── Supplementary Figure S2: Initial PyMOL renders ───────────────────────

\newpage

\begin{figure}[H]
  \centering
  \begin{subfigure}[t]{0.38\textwidth}
    \includegraphics[width=\textwidth]{figS2a_L257R_pymol_v1.png}
    \caption{L257R --- Hydrophobic core}
  \end{subfigure}\hfill
  \begin{subfigure}[t]{0.38\textwidth}
    \includegraphics[width=\textwidth]{figS2b_V157D_pymol_v1.png}
    \caption{V157D --- Hydrophobic core}
  \end{subfigure}

  \smallskip
  \begin{subfigure}[t]{0.38\textwidth}
    \includegraphics[width=\textwidth]{figS2c_R248P_pymol_v1.png}
    \caption{R248P --- DNA minor groove}
  \end{subfigure}\hfill
  \begin{subfigure}[t]{0.38\textwidth}
    \includegraphics[width=\textwidth]{figS2d_C176R_pymol_v1.png}
    \caption{C176R --- Zinc coordination}
  \end{subfigure}

  \smallskip
  \begin{subfigure}[t]{0.38\textwidth}
    \includegraphics[width=\textwidth]{figS2e_R280I_pymol_v1.png}
    \caption{R280I --- DNA major groove}
  \end{subfigure}\hfill
  \begin{subfigure}[t]{0.38\textwidth}
    \includegraphics[width=\textwidth]{figS2f_overview_pymol_v1.png}
    \caption{Overview --- All five sites}
  \end{subfigure}

  \caption{\textbf{Supplementary Figure~S2: Initial PyMOL ray-traced renders
  of the top five TP53 VUS (single-residue view).}
  Generated via a scripted \texttt{.pml} pipeline from PDB 1TUP.  Each panel
  shows the wild-type residue (green sticks) in context: protein cartoon
  (gray), DNA (orange), Zn$^{2+}$ (slate spheres).  Neighbours within
  4.0\,\AA{} shown as light-blue sticks; DNA contacts as red dashes; zinc
  coordination as slate dashes.  These renders show only the wild-type
  residue, in contrast to Fig.~\ref{fig:renders} which overlays both
  wild-type and mutant rotamers.
  All panels ray-traced at 2{,}400$\times$1{,}800 pixels, 300\,DPI.
  (a)~L257R buried in the hydrophobic core.
  (b)~V157D in the $\beta$-sandwich interior.
  (c)~R248P with DNA minor groove contacts visible.
  (d)~C176R with zinc coordination bonds.
  (e)~R280I with DNA major groove contacts.
  (f)~Overview of all five variant positions as coloured spheres on the
  full p53--DNA complex.}
  \label{fig:supp_pymol}
\end{figure}

\end{document}
